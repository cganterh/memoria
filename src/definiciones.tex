\newglossaryentry{frontend}{
    name={front-end},
    description={
        parte del software que está diseñada para
        interactuar con usuarios
    }
}

\newglossaryentry{os}{
    name={sistema operativo},
    plural={sistemas operativos},
    description={
        es un programa o conjunto de programas de un sistema
        informático que gestiona los recursos de hardware y
        provee servicios a los programas de aplicación,
        ejecutándose en modo privilegiado respecto de los
        restantes
    }
}

\newglossaryentry{ram}{
    name={memoria RAM},
    plural={memorias RAM},
    description={
        se utiliza como memoria de trabajo de computadoras
        para el sistema operativo, los programas y la mayor
        parte del software
    }
}

\newglossaryentry{navegador}{
    name={navegador},
    plural={navegadores},
    description={
        es un software, aplicación o programa que permite el
        acceso a la web
    }
}

\newglossaryentry{web}{
    name={Web},
    plural={Webs},
    description={
        es un espacio de información de código abierto donde
        los documentos y otros recursos web se identifican
        mediante direcciones URL, son vinculados entre sí
        por enlaces de hipertexto, y se pueden acceder a
        través de Internet
    }
}

\newglossaryentry{spotify}{
    name={Spotify},
    description={
        es una aplicación empleada para la reproducción de
        música vía streaming
    }
}

\newglossaryentry{netflix}{
    name={Netflix},
    description={
        es una empresa de entretenimiento que proporciona
        streaming multimedia (principalmente, películas y
        series de televisión) bajo demanda por Internet
    }
}

\newglossaryentry{atom}{
    name={Atom},
    description={
        es un editor de texto y código fuente desarrollado
        por GitHub
    }
}

\newglossaryentry{gnomeshell}{
    name={GNOME Shell},
    description={
        es la interfaz de usuario básica del entorno de
        escritorio GNOME
    }
}

\newglossaryentry{facebook}{
    name={Facebook},
    description={
        es un sitio web de redes sociales
    }
}

\newglossaryentry{youtube}{
    name={YouTube},
    description={
        es un sitio web en el cual los usuarios pueden subir
        y compartir vídeos
    }
}

\newglossaryentry{framework}{
    name={framework},
    plural={frameworks},
    description={
        es una estructura conceptual y tecnológica que sirve
        de base para la organización y desarrollo de
        software
    }
}

\newglossaryentry{nfc}{
    name={NFC},
    description={
        es una tecnología de comunicación inalámbrica, de
        corto alcance y alta frecuencia que permite el
        intercambio de datos entre dispositivos
    }
}

\newglossaryentry{qr}{
    name={QR},
    description={
        es la marca comercial para un tipo de código de
        barras bidimensional
    }
}

\newglossaryentry{lud}{
    name={ludificación},
    plural={ludificaciónes},
    description={
        es el uso de técnicas, elementos y dinámicas propias
        de los juegos y el ocio en actividades no
        recreativas con el fin de potenciar la motivación,
        así como de reforzar la conducta para solucionar un
        problema u obtener un objetivo
    }
}
