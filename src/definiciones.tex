
\newglossaryentry{cliente}{
    name={cliente},
    description={
        es un computador que consume un servicio remoto en
        otro computador conocido como servidor, normalmente
        a través de una red de telecomunicaciones
    }
}

\newglossaryentry{servidor}{
    name={servidor},
    plural={servidores},
    description={
        es un computador capaz de atender las peticiones de
        un cliente y devolverle una respuesta en
        concordancia
    }
}

\newglossaryentry{frontend}{
    name={front-end},
    description={
        parte del software que está diseñada para
        interactuar con usuarios
    }
}

\newglossaryentry{backend}{
    name={back-end},
    description={
        parte del software que procesa las entradas de
        usuario desde el front-end
    }
}

\newglossaryentry{os}{
    name={sistema operativo},
    plural={sistemas operativos},
    description={
        es un programa o conjunto de programas de un sistema
        informático que gestiona los recursos de hardware y
        provee servicios a los programas de aplicación,
        ejecutándose en modo privilegiado respecto de los
        restantes
    }
}

\newglossaryentry{ram}{
    name={memoria RAM},
    plural={memorias RAM},
    description={
        se utiliza como memoria de trabajo de computadoras
        para el sistema operativo, los programas y la mayor
        parte del software
    }
}

\newglossaryentry{navegador}{
    name={navegador},
    plural={navegadores},
    description={
        es un software, aplicación o programa que permite el
        acceso a la web
    }
}

\newglossaryentry{web}{
    name={Web},
    plural={Webs},
    description={
        es un espacio de información de código abierto donde
        los documentos y otros recursos web se identifican
        mediante direcciones URL, son vinculados entre sí
        por enlaces de hipertexto, y se pueden acceder a
        través de Internet
    }
}

\newglossaryentry{spotify}{
    name={Spotify},
    description={
        es una aplicación empleada para la reproducción de
        música vía streaming
    }
}

\newglossaryentry{netflix}{
    name={Netflix},
    description={
        es una empresa de entretenimiento que proporciona
        streaming multimedia (principalmente, películas y
        series de televisión) bajo demanda por Internet
    }
}

\newglossaryentry{atom}{
    name={Atom},
    description={
        es un editor de texto y código fuente desarrollado
        por GitHub
    }
}

\newglossaryentry{gnomeshell}{
    name={GNOME Shell},
    description={
        es la interfaz de usuario básica del entorno de
        escritorio GNOME
    }
}

\newglossaryentry{facebook}{
    name={Facebook},
    description={
        es un sitio web de redes sociales
    }
}

\newglossaryentry{youtube}{
    name={YouTube},
    description={
        es un sitio web en el cual los usuarios pueden subir
        y compartir vídeos
    }
}

\newglossaryentry{framework}{
    name={framework},
    plural={frameworks},
    description={
        es una estructura conceptual y tecnológica que sirve
        de base para la organización y desarrollo de
        software
    }
}

\newglossaryentry{nfc}{
    name={NFC},
    description={
        es una tecnología de comunicación inalámbrica, de
        corto alcance y alta frecuencia que permite el
        intercambio de datos entre dispositivos
    }
}

\newglossaryentry{qr}{
    name={QR},
    description={
        es la marca comercial para un tipo de código de
        barras bidimensional
    }
}

\newglossaryentry{lud}{
    name={ludificación},
    plural={ludificaciónes},
    description={
        es el uso de técnicas, elementos y dinámicas propias
        de los juegos y el ocio en actividades no
        recreativas con el fin de potenciar la motivación,
        así como de reforzar la conducta para solucionar un
        problema u obtener un objetivo
    }
}

\newglossaryentry{lpolling}{
    name={long polling},
    description={
        es una variación de la técnica tradicional de
        polling y permite emular información colocada desde
        un servidor a un cliente
    }
}

\newglossaryentry{ws}{
    name={WebSocket},
    description={
        es una tecnología compatible con servidores HTTP que
        proporciona un canal de comunicación bidireccional y
        full-duplex sobre un único socket TCP
    }
}

\newglossaryentry{callback}{
    name={callback},
    plural={callbacks},
    description={
        es una pieza de código ejecutable que se pasa como
        argumento a otro código, que espera para ejecutar el
        argumento en algún momento conveniente
    }
}

\newglossaryentry{futuro}{
    name={futuro},
    description={
        es un remplazo para un resultado que todavía no está
        disponible, generalmente debido a que su cómputo
        todavía no ha terminado, o su transferencia por la
        red no se ha completado
    }
}

\newglossaryentry{nbloq}{
    name={no bloqueante},
    description={
        una operación de entrada/salida no bloqueante es una
        forma de procesamiento de entrada/salida que permite
        que otros procesos puedan continuar antes de que la
        transmisión haya terminado
    }
}

\newglossaryentry{asyncio}{
    name={asyncio},
    description={
        es un módulo de Python que proporciona una
        infraestructura para escribir código concurrente de
        un único thread usando corrutinas,
        multiplexación del acceso de entrada/salida a
        sockets y otros recursos, ejecutar clientes y
        servidores de red y otras operaciones primitivas
        relacionados
    }
}

\newglossaryentry{nosql}{
    name={NoSQL},
    description={
        es una amplia clase de sistemas de gestión de bases
        de datos que difieren del modelo relacional clásico
        en aspectos importantes, el más destacado es que no
        usan SQL como el principal lenguaje de consultas
    }
}

\newglossaryentry{rdb}{
    name={base de datos relacional},
    plural={bases de datos relacionales},
    description={
        es un tipo de base de datos que cumple con el modelo
        relacional, permitiendo establecer interconexiones o
        relaciones entre los datos que son guardados en
        tablas
    }
}

\newglossaryentry{json}{
    name={JSON},
    description={
        del inglés "JavaScript Object Notation", es un
        formato de intercambio de datos ligero, fácil de
        leer y escribir para los seres humanos y fácil de
        analizar y generar para las máquinas
    }
}

\newglossaryentry{html}{
    name={HTML},
    description={
        del inglés "HyperText Markup Language", es un
        lenguaje de marcado para la elaboración de
        documentos y aplicaciones web
    }
}

\newglossaryentry{html5}{
    name={HTML5},
    description={
        quinta versión del lenguaje de marcado HTML
    }
}

\newglossaryentry{css}{
    name={CSS},
    description={
        del inglés "Cascading Style Sheets", es un lenguaje
        usado para definir la presentación de un documento
        escrito en HTML o XML
    }
}

\newglossaryentry{css3}{
    name={CSS3},
    description={
        tercera versión del lenguaje para hojas de estilo
        CSS
    }
}

\newglossaryentry{js}{
    name={JavaScript},
    description={
        es un lenguaje de programación interpretado,
        dialecto del estándar ECMAScript
    }
}

\newglossaryentry{mixin}{
    name={mixin},
    description={
        en Sass, los mixins permiten definir estilos que
        pueden ser reutilizados, sin necesidad de recurrir a
        clases no semánticas
    }
}

\newglossaryentry{c}{
    name={C},
    description={
        es un lenguaje de programación originalmente
        desarrollado por Dennis M. Ritchie entre 1969 y 1972
        en los Laboratorios Bell, como evolución del
        anterior lenguaje B
    }
}

\newglossaryentry{cpp}{
    name={C++},
    description={
        es un lenguaje de programación diseñado a mediados
        de los años 1980 por Bjarne Stroustrup. La intención
        de su creación fue el extender al lenguaje de
        programación C mecanismos que permiten la
        manipulación de objetos
    }
}
