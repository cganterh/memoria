\chapter{Contribuciones del Presente
Trabajo}\label{contribuciones-del-presente-trabajo}

\section{Concurso de Investigación
OEA}\label{concurso-de-investigaciuxf3n-oea}

Con este proyecto se ha postulado al \emph{Concurso de Proyectos de
Investigación Educativa en Ingeniería y Ciencias, Olivier Espinosa
Aldunate}. Ganando el concurso en su versión 2015.

Los fondos adjudicados por este concurso han permitido contratar a un
programador adicional que podrá continuar con el desarrollo del proyecto
durante el año 2016.

Además, se destaca la colaboración con el departamento de física. El
cual ha apoyado durante el desarrollo de la aplicación ofreciendo
retroalimentación y colaborará permitiendo realizar pruebas de la
aplicación en clases reales, realizando mediciones del impacto de estas
tecnologías en los alumnos.

\section{Aportes a proyectos
externos}\label{aportes-a-proyectos-externos}

Durante el desarrollo de la aplicación se han encontrado dos pequeños
errores en proyectos de código abierto externos. Para poder continuar
con el desarrollando, se ofrecieron soluciones a ambos problemas, siendo
estas aceptadas por las respectivas comunidades de desarrolladores.

Los proyectos en los que se encontraron estos errores fueron Bourbon y
httplib2. Ambos son proyectos grandes, de bastante impacto y con
comunidades internacionales de desarrolladores. En el caso de httplib2,
esta es mantenida por trabajadores de Google y es dependencia de
oauth2client, la biblioteca para la autenticación de usuarios de Google
en Python.

\section{Módulos de software
reutilizables}\label{muxf3dulos-de-software-reutilizables}

Como ya se ha mencionado en secciones anteriores, el código de
TornadoBoiler y TornadoBoxes (bases para la aplicación desarrollada en
este trabajo) es completamente reutilizable. Pero además, existen
módulos dentro de la aplicación que podrían ser empaquetados y
distribuidos de manera independiente.

Este es el caso, por ejemplo, del módulo que implementa los adaptadores
PubSub. PubSub es un patrón de diseño común que, si bien tiene algunas
implementaciones en Python, ninguna de ellas funciona con el bucle de
eventos que provee Tornado. Como Tornado es una librería que puede
manejar conexiones a múltiples recursos de red, el módulo PubSub sería
un buen candidato para implementar este patrón de diseño en otros
proyectos basados en Tornado.

Otro ejemplo, que requeriría un poco mas de trabajo para ser empaquetado
es la clase de autenticación con Google. Las clases que provee Tornado
para autenticar usuarios con Google, están desactualizadas. Incluso los
mismos desarrolladores de Tornado admiten en foros que los mecanismos de
autenticación con Google no están siendo mantenidos y no han sido
probados extensivamente. Es por eso que en este trabajo se han tenido
que implementar estos mecanismos desde cero.

\section{Liberación del código}\label{liberaciuxf3n-del-cuxf3digo}

Como ya ha sido mencionado, este es un proyecto de código abierto. Lo
cual significa que el código está disponible para la comunidad sin
ningún tipo de restricción de copia o modificación. Esto suele ser
beneficioso para este tipo de proyectos, ya que aumenta la probabilidad
de que el código sea mantenido y de que se forme una comunidad alrededor
del proyecto. Finalmente, beneficiando a la comunidad.
