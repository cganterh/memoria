\chapter{Contribuciones del Presente
Trabajo}\label{contribuciones-del-presente-trabajo}

\section{Proyecto de Investigación OEA
Adjudicado}\label{proyecto-de-investigaciuxf3n-oea-adjudicado}

Con este proyecto se ha postulado al \emph{Concurso de Proyectos de
Investigación Educativa en Ingeniería y Ciencias, Olivier Espinosa
Aldunate}. Ganando el concurso en su versión 2015.

Los fondos adjudicados por este concurso han permitido contratar a un
programador adicional que podrá continuar con el desarrollo del proyecto
durante el año 2016.

Además, se agradece la colaboración con el departamento de física. El
cual ha brindado su apoyo durante el desarrollo de la aplicación
ofreciendo retroalimentación. Y colaborará permitiendo realizar pruebas
de la aplicación en clases reales, realizando mediciones del impacto de
estas tecnologías en los alumnos.

\section{Aportes a Proyectos
Externos}\label{aportes-a-proyectos-externos}

Bourbon y httplib2 son dos proyectos de código abierto grandes, de
bastante impacto y con comunidades internacionales de desarrolladores.
En el caso de httplib2, este es mantenido por trabajadores de Google y
es dependencia de oauth2client, la biblioteca para la autenticación de
usuarios de Google en Python. Durante el desarrollo de la aplicación se
han encontrado pequeños errores en estos proyectos externos. Para poder
continuar con el desarrollo, se ofrecieron soluciones a ambos problemas,
siendo estas aceptadas por las respectivas comunidades.

\section{Módulos de Software
Reutilizables}\label{muxf3dulos-de-software-reutilizables}

Como ya se ha mencionado en secciones anteriores, el código de
TornadoBoiler y TornadoBoxes (bases para la aplicación desarrollada en
este trabajo) es completamente reutilizable. Pero además, existen
módulos dentro de la aplicación que podrían ser empaquetados y
distribuidos de manera independiente.

Este es el caso, por ejemplo, del módulo que implementa los adaptadores
PubSub. PubSub es un patrón de diseño común que, si bien tiene algunas
implementaciones en Python, ninguna de ellas funciona con el bucle de
eventos que provee Tornado. Como Tornado es una librería que puede
manejar conexiones a múltiples recursos de red, el módulo PubSub sería
un buen candidato para implementar este patrón de diseño en otros
proyectos basados en Tornado.

Otro ejemplo, que requeriría un poco más de trabajo para ser empaquetado
es la clase de autenticación con Google. Las clases que provee Tornado
para autenticar usuarios con Google, están desactualizadas. Incluso los
mismos desarrolladores de Tornado admiten en foros que los mecanismos de
autenticación con Google no están siendo mantenidos y no han sido
probados extensivamente. Es por eso que en este trabajo se ha tenido que
implementar estos mecanismos desde cero.

\section{Liberación del Código}\label{liberaciuxf3n-del-cuxf3digo}

Como ya ha sido mencionado este es un proyecto de código abierto, lo
cual significa que el código está disponible para la comunidad sin
ningún tipo de restricción de copia o modificación. Esto suele ser
beneficioso para este tipo de proyectos, ya que aumenta la probabilidad
de que el código sea mantenido y de que se forme una comunidad alrededor
del proyecto.

Para este proyecto se ha escogido la licencia de software libre
\href{http://www.gnu.org/licenses/agpl-3.0.html}{\emph{GNU Affero
General Public License}}. Esta es un derivado de la licencia \emph{GNU
General Public License}. Ambas licencias permiten el uso comercial, la
distribución y la modificación del software. Además, requieren que la
distribución de trabajos derivados se haga con la misma licencia que el
trabajo original. Cualquier copia distribuida bajo estas licencias debe
incluir el código fuente del software.

La diferencia entre \emph{GNU Affero General Public License} y \emph{GNU
General Public License} radica en que, con la segunda, una persona puede
hacer una modificación del software y usarla de manera privada sin
liberar el código. Esto normalmente no es un problema. Sin embargo,
cuando el software en cuestión está hecho para ejecutarse en un
\gls{servidor}, la persona que modificó el software podría sacar
provecho de sus modificaciones sin publicarlas. En este caso el software
no está siendo distribuido, pero la persona se beneficia de su
modificación sin contribuir a la comunidad. Debido a este detalle la
licencia \emph{GNU Affero General Public License} incluye una condición
que especifica que, cuando un software publicado bajo esta licencia se
ejecuta en un \gls{servidor} para prestar servicios, se debe proveer un
mecanismo para que los usuarios descarguen el código fuente.
